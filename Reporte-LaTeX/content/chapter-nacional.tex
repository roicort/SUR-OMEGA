\chapter{Contexto Nacional}

Desde 1975 a 2019 Colombia ha experimentado un crecimiento sin precedentes. Durante este la población colombiana se ha duplicado, pasando de aproximadamente 24 a 49 millones de habitantes, por otro lado, la industria y el comercio también han crecido, el Producto Interno Bruto ha aumentado 4.8 veces en este mismo periodo. Estos cambios muestran el cambio de un Colombia poco comunicado e industrializado a uno más urbanizado y moderno. No obstante, este crecimiento también ha implicado un aumento más que significativo en el consumo de energía y en la composición de la oferta de energéticos. Cambios que implican nuevos desafíos así como áreas de oportunidad para el gobierno y la sociedad colombiana. 

En el sector transporte se ha presentado una tasa promedio anuales de crecimiento del  5,9\%. \cite{Plan_Energetico-2050} y se estima que en Colombia las emisiones provenientes de este sector representan el 10\% de las emisiones de gases de efecto invernadero a nivel nacional, siendo el transporte terrestre el componente más importante (vehículos ligeros y de carga) (IDEAM,2016). \cite{InformeFinal_WRI}

\section{Plan Energético Nacional 2020-2050}

Recientemente la Unidad de Planeación Minero-Energética (UPME) presentó el Plan Energético Nacional 2020-2050: La transformación energética que habilita el desarrollo sostenible. En este documento se analizan las políticas y metas ya establecidas en materia energética, el impacto de medidas más ambiciosas y los desafíos relacionados con la adopción de tecnologías ya comerciales y otras que aún se encuentran en desarrollo.

Dentro del Plan Energético Nacional 2020-2050, podemos encontrar uno de los principales retos en materia energética que será abastecer una demanda creciente de energía durante los próximos 30 años. En este escenario el sector transporte será el gran protagonista, los potenciales ahorros de energía que se pueden alcanzar con la adopción de mejores tecnologías para utilizar menos combustibles fósiles o fuentes alternativas de energía, hacen del sector transporte (ver \ref{fig:pn-participacion})el principal actor del país en términos de cambio energético y mitigación del cambio climático en este escenario. \cite{Plan_Energetico-2050}

\begin{figure}[htbp]
   \centering
   \includesvg[width=1\textwidth]{figures/pn-participacion.svg}
    \caption{Participación por sectores en el consumo final de energía (PJ) 2019 \cite{Plan_Energetico-2050}}
    \label{fig:pn-participacion}
\end{figure}


\section{Desafíos}

\subsection{Disponibilidad de Recursos}

Hasta ahora, Colombia ha podido abastecer su demanda de energía en mayor parte con recursos internos. No obstante, las tendencias a largo plazo de oferta y demanda indican que la autosuficiencia energética podría terminar. En el \textit{Plan indicativo de combustibles líquidos} de la UPME en 2018, se encuentra que se necesitaría importar crudos livianos y pesados en 2028 para cumplir con los requerimientos de las refinerías y la demanda, en caso de que se presentaran las condiciones del escenario de baja producción de petróleo.


\subsection{Brecha Tecnológica}

La eficiencia energética ofrece una de las mayores áreas de oportunidad para cumplir con otros indicadores ambientales, reducir costos de producción en el país y aumentar la competitividad de las empresas. Por tanto, el segundo reto del sector energético colombiano es reducir la brecha tecnológica para aumentar la eficiencia energética.

 \begin{figure}[htbp]
   \centering
   \includesvg[width=1\textwidth]{figures/pn-sectores.svg}
    \caption{Energía final con la mejor tecnología disponible nacional e internacional (UPME, 2018)}
    \label{fig:pn-sectores}
\end{figure}

En el Balance de Energía Útil para Colombia (BEU) se compara la energía que consumen las tecnologías que se usan actualmente en el país, con respecto a la que se consumiría si se utilizaran las mejores tecnologías disponibles a escalas nacional e internacional, denominadas Best Availa- ble Technologies (BAT). En Colombia la energía útil es apenas el 31\% de la final y la ineficiencia en el consumo es del orden del 67\%, situación que le cuesta anualmente al país entre 6.600 y 11.000 millones de USD al año. Dentro de este porcentaje, el sector transporte representa la mayor contribución a este problema. Según BEU, el consumo de energía final del país se puede reducir entre un 38\% y un 50\% con el cambio en todas las tecnologías del sector en BAT Nacional y hasta 62\% en BAT Internacional, siendo en ambos casos, principalmente aquellas tecnologías aplicables en el sector transporte.

El BEU indica que la energía útil en el sector transporte es solo el 24\% de la energía que se consume y las pérdidas por equipamiento corresponden al 69\%. El potencial de mejora de eficiencia para este sector, si se adoptaran las BAT nacionales sería del 50\%, lo que representaría un ahorro del orden de los 3.400 millones de USD al año si se adoptan las BAT Nacionales y de 6.000 millones de USD al año si se adoptan BAT Internacionales.

 \begin{figure}[htbp]
   \centering
   \includesvg[width=0.7\textwidth]{figures/pn-transporte.svg}
    \caption{Energía final y energía útil del sector transporte (UPME, 2018)}
    \label{fig:pn-transporte}
\end{figure}

\subsection{Mitigación y adaptación al cambio climático}

El cambio climático es un fenómeno que pone en riesgo la existencia de la vida humana, la biodiversidad y los ecosistemas. Según el 5° Informe de Evaluación del Grupo Intergubernamental de Expertos sobre el Cambio Climático (IPCC) se concluyó que «es extremadamente probable que la influencia humana ha sido la causa dominante del calentamiento observado desde la mitad del siglo XX» y que esta influencia parta de la emisión de gases de efecto invernadero como el dióxido de carbono, el metano y el óxido de nitrógeno. Las actividades humanas desde el inicio de la Revolución Industrial (desde 1750) ha producido un incremento del 40\% en la concentración atmosférica del dióxido de carbono (CO2), desde 280 partes por millón (ppm) en 1750 a 400 ppm en 2015. Este incremento ha ocurrido a pesar de la absorción de una gran porción de las emisiones por varios depósitos naturales que participan del ciclo del carbono. Las emisiones de CO2 antropogénicas (producidas por actividades humanas) provienen de la combustión de combustibles fósiles, principalmente carbón, petróleo y gas natural, además de otras actividades que agravan el problema, como la deforestación, la erosión del suelo y la crianza animal. En el último siglo, sin embargo, el registro de temperaturas desde la década de 1950 no tiene precedentes (ni en el registro de las temperaturas históricas ni en el de los estudios paleoclimáticos). \label{Manual de Argumentación}

La Organización de Naciones Unidas ha calificado al cambio climático como el “mayor desafío de nuestro tiempo”, debido a que sus efectos trascienden fronteras y generaciones y por ello se requiere una respuesta coordinada de todos los países para frenarlo. \cite{Plan_Energetico-2050}

En términos energéticos, las medidas de mitigación ante el cambio climático pasan por la adopción de mejoras de eficiencia energética que permitan reducir el consumo y la sustitución de combustibles fósiles por fuentes de bajas emisiones. En este contexto, Colombia se ha comprometido a reducir sus emisiones de Gases de Efecto Invernadero (GEI). \cite{Plan_Energetico-2050} Como se mencionaba anteriormente, es preciso que Colombia acelere la adopción de tecnologías para mejorar la eficiencia en el consumo energético, por lo que las medidas tendientes a reducir el consumo a través de eficiencia energética sirven tanto para mejorar la competitividad como para mitigar los efectos del cambio climático.



\section{Normatividad}

En la actualidad, Colombia no cuenta con un marco normativo que monitoree y controle la eficiencia energética en los vehículos. En este sentido se está trabajando en el desarrollo de la hoja de ruta para la implementación de los vehículos de bajas y cero emisiones en el país, así como para establecer la normativa de eficiencia energética y etiquetado vehicular en los diferentes segmentos del transporte. \cite{Baseline_Pereira_2020}

Adicionalmente, el país cuenta con una serie de programas regionales y nacionales: \label{Comentario}[Adjuntar resoluciones?]

\section{Línea Base 2017}

En el año 2017, en el marco del proyecto “Combustibles y Vehículos más Limpios y Eficientes” se estableció la línea base de consumo de combustible para los vehículos livianos (LDV) de Colombia, considerando los años 2011, 2012, 2014 y 2016. \cite{Baseline_Pereira_2020}

A partir la base de datos consolidada se pudo concluir que, para Colombia, el factor de emisión anual promedio de CO2, calculado bajo el ciclo de conducción NEDC, presentó una reducción del 7,77\% entre el 2011 y el 2016, pasando de 175,81 gCO2/km a 163,13 gCO2/km, respectivamente. Así mismo, se pudo observar una mejora en el rendimiento de combustible de la flota ya que el consumo de combustible se redujo de 7,39 lge/100 km en 2011 a 6,98lge/100km en 2016.\cite{Baseline_Pereira_2020}

\section{Línea Base 2020}

En el año 2020, la Universidad Tecnológica de Pereira en conjunto con el Programa de las Naciones Unidas para el Medio Ambiente (PNUMA), socios de la Iniciativa Global para la Economía de Combustible (GFEI) y muchas otras mas entidades, se estableció la línea base para la economía de combustible de los vehículos livianos (LDV) en Colombia del año 2017, frente a los resultados de la actualización del estudio hasta el año modelo 2019. Se realizó una búsqueda de información relacionada principalmente con:

\begin{itemize}
\item Calidad de combustible
\item Emisiones vehiculares
\item Eficiencia energética en el sector transporte
\item Caracterización de la flota de vehículos pesados (HDV)
\item Planes o proyectos a futuro relacionados con los temas investigados en los ítems
anteriores.
\item Identificación de los órganos competentes en materia de medio ambiente, energía y
transporte.
\end{itemize}

El presente informe presenta los resultados de la actualización de la línea base de consumo de combustible [lge/100km] y de dióxido de carbono [gCO2/km] desarrollado para los vehículos livianos (LDV) que ingresaron al mercado colombiano bajo año modelo 2017, 2018 y 2019. Las metodologías utilizadas corresponden a las desarrolladas por la Iniciativa Global para la Economía de Combustibles (GFEI), las cuales han sido implementadas en diferentes países, permitiendo que los resultados obtenidos en el contexto colombiano sean comparables con los de otros estudios realizados alrededor del mundo. \cite{Baseline_Pereira_2020}

\section{Parque Vehicular en Colombia}

De acuerdo con la base de datos del Registro Único Nacional de Transito – RUNT 2020 se realizó un análisis de la flota de vehículos que opera en el territorio colombiano de acuerdo a tres criterios de participación: clase, combustible y marca. Los porcentajes corresponden a cifras agregadas hasta el año 2019 según año modelo.

 \begin{figure}[htbp]
   \centering
   \includesvg[width=1\textwidth]{figures/linebase_clase.svg}
    \caption{Distribución del parque automotor colombiano por clase (2019) \cite{ICCT_Performance_2020}}
    \label{fig:linebaseclase}
\end{figure}

En cuanto a la participación por clase, podemos observar que \ref{fig:linebaseclase} muestra que el automóvil corresponde a la tipología de vehículos con mayor participación con un 59.3\%, seguido de la camioneta con un 34.6\%. De acuerdo al tipo de combustible, en \ref{fig:linebasecomb} indica que los vehículos que operan con gasolina representan el 92.1\%, mientras que los vehículos diésel representan el 7.3\% de la flota de vehículos livianos.

 \begin{figure}[htbp]
   \centering
   \includesvg[width=1\textwidth]{figures/linebase_comb.svg}
    \caption{Distribución del parque automotor colombiano por tipo de combustible. \cite{ICCT_Performance_2020}}
    \label{fig:linebasecomb}
\end{figure}

Es también importante tener en cuenta que el 70\% distribución del parque vehicular colombiano por marcas de casas fabricantes de vehículos esta concentrada solo en seis marcas. De estas marcas, la mayoría de los vehículos importado provienen de países que ya cuentan con políticas para la eficiencia energética, por ejemplo México. 