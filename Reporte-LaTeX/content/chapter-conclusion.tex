\chapter{Conclusiones}

Implementar mecanismos que garantizan la eficiencia energética, sobre todo en combustibles fósiles, implica realizar cambios productivos, estructurales y de consumo que tendrán un impacto que trasciende lo únicamente lo energético. La reducción del uso de combustibles fósiles tendrá repercusión en la economía del país, el empleo, los recursos con los que cuentan las regiones, la forma de vida de los habitantes, el medio ambiente en el que se explotan recursos, etc. Por ello, el planeamiento energético debe involucrar tales impactos en los análisis e involucrar a los diferentes actores que podrían intervenir en el proceso. Analizando los retos que Colombia deberá seguir enfrentando durante los próximos años es notable que el sector transporte y la mejora de la eficiencia energética es un factor clave para garantizar el cumplimiento de los objetivos que permitan mitigar los efectos de dichos retos y asegurar el futuro económico, tecnológico y ambiental del país.

El Modelo OMEGA permite observar la viabilidad de la implementación de tecnologías que permitan a los fabricantes cumplir con las metas de eficiencia así como cerrar la brecha tecnológica.

Es posible identificar nuevas áreas de mejora y trabajo futuro. A pesar de que el modelo OMEGA es robusto, desde la salida inicial del Modelo OMEGA I distintos avances tecnológicos y cambios en el mercado de Vehículos Livianos (LDV) eventualmente han surgido así como nuevos servicios de movilidad. La versión de OMEGA II se desarrolló tomando en cuenta todo estos nuevos cambios. La introducción de la interacción de decisiones entre Consumidores y Productores permite al modelo representar la respuesta de los consumidores a nuevos vehículos y servicios.

No obstante debido a que el modelo OMEGA II considera nuevas interacciones, la cantidad de datos necesarios para modelarlas es mayor. Considerando la dificultad de obtener datos específicos para Colombia este se trata de uno de los retos principales pero también la mayor área de oportunidad. Es importante para los gobiernos capturar datos que permitan utilizar modelos para extraer información de los mismos y permitan tomar decisiones sustentadas.

\begin{figure}[htbp]
   \centering
   \includesvg[width=0.75\textwidth]{figures/omega2.svg}
    \caption{OMEGAII}
    \label{fig:omega2}
\end{figure}
