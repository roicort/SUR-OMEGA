\chapter{Introducción}

Ante el panorama mundial actual y los importantes retos a enfrentar respecto a mitigación y adaptación al cambio climático, disponibilidad de recursos y brechas tecnológicas, más de 120 países se han comprometido a tomar medidas para mejorar la eficiencia en vehículos livianos. Los países latinoamericanos tienen el reto de crear, adaptar y mejorar los esfuerzos internacionales para convertirse en líderes en eficiencia de combustibles y reducción de emisiones; todo esto desde estrategias en políticas públicas y normas que contribuyan a éstas. El uso ineficiente de recursos energéticos implica una pérdida irrecuperable para los gobiernos. Comenzando por pérdidas económicas cuando se puede utilizar menos (reducir los costos) para producir lo mismo o con la misma cantidad de energía que se consume se podría producir más (aumentar productividad), hasta el olvido de la enorme área de oportunidad que existe y los beneficios para las sociedades en términos de salud y medio ambiente.

Las normas de ahorro de combustible (EF) o de emisiones de gases de efecto invernadero (GEI) son uno de los principales instrumentos de que disponen los responsables políticos para lograr mejoras significativas en el consumo de combustible y las emisiones de GEI de los vehículos ligeros (VL). Estas normas exigen que los nuevos vehículos ligeros reduzcan el consumo de combustible y las emisiones de gases de efecto invernadero a lo largo del tiempo mediante el desarrollo y la aplicación continuos de tecnologías de ahorro de combustible. La adopción de estas normas da lugar a una transformación del mercado hacia vehículos cada vez más eficientes en el consumo de combustible, que consumen menos combustible por kilómetro recorrido y, por tanto, emiten menos GEI. \cite{FESET_ICCT}
El éxito de la aplicación de las nuevas normas de GEI/FE de los vehículos se traduce en la incorporación de vehículos más eficientes a la flota, lo que, combinado con la retirada natural de los modelos más antiguos y menos eficientes, se traduce en una mejora de la eficiencia media del combustible de la flota nacional. Las entidades reguladas son los fabricantes de vehículos y los importadores de todos los vehículos nuevos destinados a la venta en el país. Cada fabricante de automóviles debe cumplir un valor objetivo basado en el parque de vehículos de baja cilindrada que vende.Los fabricantes pueden elegir la vía tecnológica más adecuada a su plan de negocio, respetando las preferencias de los consumidores locales.

Una de las maneras para alcanzar las metas establecidas de eficiencia en el consumo de combustible, es la implementación de tecnologías para este fin en el vehículo. La aplicación de paquetes tecnológicos resulta viable, dado el desarrollo acelerado en las tecnologías para la reducción en el consumo de combustible y para la reducción de emisiones de gases de efecto invernadero y contaminantes criterio. El objetivo de este trabajo es la aplicación del Modelo OMEGA al mercado Colombiano para obtener una comprensión de las tecnologías necesarias para incrementar la eficiencia en vehículos livianos así como de los costos y beneficios de los diferentes escenarios regulatorios para la siguiente fase de las normas de GHG para vehículos en Colombia.

Como parte de las metas nacionales de mitigación del cambio climático, en las que Colombia se compromete a reducir el 20\% de sus emisiones de dióxido de carbono ($CO_2$) para el 2030 respecto a las emisiones proyectadas (Arbeláez et al., 2015), y teniendo en cuenta que las fuentes móviles representan uno de los sectores en donde se produce una importante cantidad de emisiones $CO_2$, se plantea la necesidad de una regulación para acceder a vehículos más eficientes en el consumo de combustible. Este tipo de regulación es justificada por la Agencia de Protección Ambiental de Estados Unidos, quien argumenta que es complicado que los fabricantes tengan voluntad propia para actualizar la tecnología de los vehículos para incrementar el redimiento de combustible, es decir, es necesario que el gobierno imponga límites mínimos de rendimiento de combustible, para que los fabricantes actualicen la tecnología de los vehículos sustancialmente \cite{EPA_2012-A}. Se estima que en Colombia las emisiones provenientes del transporte representan el 11\% de las emisiones de gases de efecto invernadero a nivel nacional, siendo el transporte terrestre el componente más importante (vehículos ligeros y de carga) (IDEAM, 2016). \label{Informe Final}

Según el Departamento Nacional de Planeación del Gobierno de Colombia “Los costos en la salud asociados a la degradación ambiental en Colombia ascienden a \$20,7 billones de pesos (más de 7 mil millones de dólares), equivalentes al 2,6\% del PIB del año 2015, relacionados con 13.718 muertes y cerca de 98 millones de síntomas y enfermedades. Dentro de estos costos, la contaminación del aire urbano aportó el 75\%, con \$15,4 billones de pesos (1,93\% del PIB de 2015) asociados a 10.527 muertes y 67,8 millones de síntomas y enfermedades. Recordemos, además, que la contaminación del aire está relacionada con muertes por Cáncer de pulmón, enfermedad cardiopulmonar, todas las causas de muerte en menores de 5 años y mortalidad general. 

En este informe haremos una revisión de los esfuerzos internacionales para la mejora de la eficiencia energética en combustibles y sobre el contexto nacional en el colombia del que parte este trabajo. Revisaremos los restos que enfrentará colombia durante los años siguientes y como la implementación de medidas que promuevan la mejora de la eficiencia energética puede mitigar los retos económicos y climáticos. Posteriormente para el fin de la modelación del mercado vehícular en Colombia analizaremos la metodología propuesta para OMEGA y los supuestos necesarios para la misma así como los datos utilizados.



