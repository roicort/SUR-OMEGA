
\chapter{Contexto Internacional}

A nivel internacional, diferentes países han establecido metas de eficiencia energética en el consumo de combustible en vehículos ligeros. Por ejemplo, Estados Unidos, a través de la Administración Nacional de Seguridad de Tráfico en Carreteras (NHTSA), estableció desde 1975 el promedio de consumo de combustible ponderado por corporación (CAFE) con el objetivo de reducir el consumo de combustible de los vehículos y tener una mejora en la eficiencia energética. En 2012, el CAFE fue actualizado para establecer los límites de consumo de combustible, así como los límites en la emisión de CO2 para la flota estadounidense. El límite para 2025 fue establecido en una meta final de consumo de combustible de 87,4 kilómetros por galón (54,3 mpg) y 101,2 gramos de CO2 por kilómetro (163 gCO2/mi) para el año 2025 (EPA, 2012 B) \cite{EPA_2012-B}. Así mismo, para el periodo de 2012 a 2016, el límite de emisión de los automóviles de Estados Unidos fue establecido en 155 gCO2/km. La Figura 1 muestra el comportamiento del consumo de combustible de la flota vehicular de Estados Unidos, contra los límites que se han impuesto. Se puede observar que el promedio de consumo de combustible ponderado por corporativo ha estado ligeramente por encima de lo requerido. Este comportamiento sucede cuando se establece la posibilidad de intercambiar créditos que son producidos cuando los fabricantes de automóviles sobrepasan la meta requerida, y cumplen con una meta más estricta que la establecida. \label{Informe Final}

\paragraph{Unión Europea}
Por otro lado, la Unión Europea (UE) comenzó con los esfuerzos para establecer un límite de eficiencia en el consumo de combustible y emisiones de CO2 desde 1998 con una meta de 140 gCO2/km alcanzable para 2008. La última actualización de la UE establece una meta de consumo de combustible de aproximadamente 91,98 km/gal y un límite de emisión de CO2 de 95 gCO2/km, para 2020 (EU, 2014).  \label{Informe Final}

\paragraph{México}
En México, los esfuerzos por tener vehículos ligeros más eficientes en el consumo de combustible concluyeron en 2013 con la publicación de la norma NOM-163-SEMARNAT-ENER-SCFI-2013 que regula las emisiones de CO2 provenientes del tubo de escape y su equivalencia en términos de rendimiento de combustible, aplicable a vehículos automotores nuevos de peso bruto vehicular de hasta 3.857 kilogramos. En esta norma se establece que para 2016, los automóviles deberán emitir desde 135 gCO2/km hasta 180 gCO2/km, dependiendo de su tamaño, y para camionetas de 163,6 gCO2/km hasta 227 gCO2/km dependiendo de su tamaño (DOF, 2013).  \label{Informe Final}

\paragraph{Japón}
De la misma manera, Japón establece un límite de consumo de combustible de 76,8 km/gal alcanzable para 2020 combustible en la política denominada “Top Runner” (AESS et al., 2011). \\ \label{Informe Final}

En el 2014, 27 de 34 países de la OECD modificaron la normatividad tributaria y realizaron exenciones para la adquisición de vehículos con alta eficiencia en el consumo de combustible y de baja emisión de contaminantes atmosféricos. Además, 18 países disminuyeron la tasa de cobro de impuestos sobre vehículos eléctricos o híbridos (Econcept - AEI, 2016). \label{Informe Final}

Una de las maneras para alcanzar la meta establecida de eficiencia en el consumo de combustible, es la implementación de tecnología en el vehículo. La aplicación de paquetes tecnológicos resulta viable, dado el desarrollo acelerado en las tecnologías para la reducción en el consumo de combustible y para la reducción de emisiones de gases de efecto invernadero y contaminantes criterio; además Colombia posee la ventaja de ser importante productor de refacciones para automóviles (Rivera Godoy et al., 2016). 

\section{GFEI}

\section{PCFV}

\section{Caso de estudio EU}

En la Unión Europea podemos observar mejoras cuantificables a partir de las normativas implementadas durante los últimos años. Los datos preeliminares de la \textit{European Environment Agency} (EEA) mostraron que los coches nuevos vendidos en la UE en 2020 tenían unas emisiones medias de $CO_2$ de $108g$ de $CO_2/km$, $14g/km$ menos que en 2019, medido sobre el Nuevo Ciclo de Conducción Europeo (NEDC). \cite{ICCT_Performance_2020} Gracias a los mecanismos de cumplimiento de la EU, el promedio de emisiones de $CO_2$ bajo el Cíclo NEDC se encuentra en $96 g/km$. El efecto de los mecanismo es contundente como podemos observar en \ref{fig:iccp-performance} y casi todos los fabricantes cumplieron sus objetivos de $CO_2$ para 2020.

 \begin{figure}[htbp]
   \centering
   \includesvg[width=1\textwidth]{figures/iccp-performance.svg}
    \caption{Valores Medios Históricos de las Emisiones de CO2 NEDC \cite{ICCT_Performance_2020}}
    \label{fig:iccp-performance}
\end{figure}