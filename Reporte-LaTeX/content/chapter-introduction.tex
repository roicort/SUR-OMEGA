\chapter{Introducción}

Ante el panorama mundial actual, en el que más de 120 países que se han comprometido a trabajar ante el avance inminente de la crisis climática, mediante diversas medidas que reduzcan las emisiones que generan; los países latinoamericanos tienen ante sí el reto de crear, adaptar y mejorar los esfuerzos para ser líderes en reducción de emisiones; desde cambios en la política pública, hasta normas que contribuyan a éstas.  \label{Estrategia de Argumentación}

Como parte de las metas nacionales de mitigación del cambio climático, en las que Colombia se compromete a reducir el 20\% de sus emisiones de dióxido de carbono ($CO_2$) para el 2030 respecto a las emisiones proyectadas (Arbeláez et al., 2015), y teniendo en cuenta que las fuentes móviles representan uno de los sectores en donde se produce una importante cantidad de emisiones $CO_2$, se plantea la necesidad de una regulación para acceder a vehículos más eficientes en el consumo de combustible. Este tipo de regulación es justificada por la Agencia de Protección Ambiental de Estados Unidos, quien argumenta que es complicado que los fabricantes tengan voluntad propia para actualizar la tecnología de los vehículos para incrementar el redimiento de combustible, es decir, es necesario que el gobierno imponga límites mínimos de rendimiento de combustible, para que los fabricantes actualicen la tecnología de los vehículos sustancialmente (EPA, 2012 A) \cite{EPA_2012-A}.
Se estima que en Colombia las emisiones provenientes del transporte representan el 11\% de las emisiones de gases de efecto invernadero a nivel nacional, siendo el transporte terrestre el componente más importante (vehículos ligeros y de carga) (IDEAM, 2016). \label{Informe Final}

Según el Departamento Nacional de Planeación del Gobierno de Colombia “Los costos en la salud asociados a la degradación ambiental en Colombia ascienden a \$20,7 billones de pesos (más de 7 mil millones de dólares), equivalentes al 2,6\% del PIB del año 2015, relacionados con 13.718 muertes y cerca de 98 millones de síntomas y enfermedades. Dentro de estos costos, la contaminación del aire urbano aportó el 75\%, con \$15,4 billones de pesos (1,93\% del PIB de 2015) asociados a 10.527 muertes y 67,8 millones de síntomas y enfermedades. Recordemos, además, que la contaminación del aire está relacionada con muertes por Cáncer de pulmón, enfermedad cardiopulmonar, todas las causas de muerte en menores de 5 años y mortalidad general. Aunque no es lo mismo la reducción de Contaminantes Criterio (CC) a la de $CO_2$ (el mayor contribuyente al cambio climático); la reducción de ambos es clave para mejorar la calidad de vida de las personas.  \label{Estrategia de Argumentación}

Por otro lado, en materia económica, el uso ineficiente de los recursos energéticos implica pérdidas económicas irrecuperables, es decir, existe una pérdida cuando se puede utilizar menos (reducir los costos) para producir lo mismo o con la misma cantidad de energía que se consume se podría producir más (aumentar productividad).

El objetivo de la aplicación del Modelo OMEGA al mercado Colombiano fue obtener una comprensión de las tecnologías necesarias para incrementar la eficiencia en vehículos livianos asi como de los costos y beneficios de los diferentes escenarios regulatorios para la siguiente fase de las normas de GHG para vehículos en Colombia.