\chapter{Conclusiones}

Implementar mecanismos que garantizan la eficiencia energética, sobre todo en combustibles fósiles, implica realizar cambios productivos, estructurales y de consumo que tendrán un impacto que trasciende lo únicamente lo energético. La reducción del uso de combustibles fósiles tendrá repercusión en la economía del país, el empleo, los recursos con los que cuentan las regiones, la forma de vida de los habitantes, el medio ambiente en el que se explotan recursos, etc. Por ello, el planeamiento energético debe involucrar tales impactos en los análisis e involucrar a los diferentes actores que podrían intervenir en el proceso.

\subsection{Áreas de Mejora}

Desde la salida inicial del Modelo OMEGA I distintos avances tecnológicos y cambios en el mercado de Vehículos Livianos (LDV) eventualmente han surgido así como nuevos servicios de movilidad. La versión de OMEGA II se desarrolló tomando en cuenta todo estos nuevos cambios. La introducción de la interacción de decisiones entre Consumidores y Productores permite al modelo representar la respuesta de los consumidores a nuevos vehículos y servicios.

\begin{figure}[htbp]
   \centering
   \includesvg[width=1\textwidth]{figures/omega2.svg}
    \caption{OMEGAII}
    \label{fig:omega2}
\end{figure}
