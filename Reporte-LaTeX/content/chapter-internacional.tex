
\chapter{Contexto Internacional}

A nivel internacional, diferentes países han establecido metas de eficiencia energética en el consumo de combustible en vehículos ligeros. Por ejemplo, Estados Unidos, a través de la Administración Nacional de Seguridad de Tráfico en Carreteras (NHTSA), estableció desde 1975 el promedio de consumo de combustible ponderado por corporación (CAFE) con el objetivo de reducir el consumo de combustible de los vehículos y tener una mejora en la eficiencia energética. En 2012, el CAFE fue actualizado para establecer los límites de consumo de combustible, así como los límites en la emisión de CO2 para la flota estadounidense. El límite para 2025 fue establecido en una meta final de consumo de combustible de 87,4 kilómetros por galón (54,3 mpg) y 101,2 gramos de CO2 por kilómetro (163 gCO2/mi) para el año 2025 (EPA, 2012 B) \cite{EPA_2012-B}. Así mismo, para el periodo de 2012 a 2016, el límite de emisión de los automóviles de Estados Unidos fue establecido en 155 gCO2/km. La Figura 1 muestra el comportamiento del consumo de combustible de la flota vehicular de Estados Unidos, contra los límites que se han impuesto. Se puede observar que el promedio de consumo de combustible ponderado por corporativo ha estado ligeramente por encima de lo requerido. Este comportamiento sucede cuando se establece la posibilidad de intercambiar créditos que son producidos cuando los fabricantes de automóviles sobrepasan la meta requerida, y cumplen con una meta más estricta que la establecida. \label{Informe Final}

\paragraph{Unión Europea}
Por otro lado, la Unión Europea (UE) comenzó con los esfuerzos para establecer un límite de eficiencia en el consumo de combustible y emisiones de CO2 desde 1998 con una meta de 140 gCO2/km alcanzable para 2008. La última actualización de la UE establece una meta de consumo de combustible de aproximadamente 91,98 km/gal y un límite de emisión de CO2 de 95 gCO2/km, para 2020 (EU, 2014).  \label{Informe Final}

\paragraph{México}
En México, los esfuerzos por tener vehículos ligeros más eficientes en el consumo de combustible concluyeron en 2013 con la publicación de la norma NOM-163-SEMARNAT-ENER-SCFI-2013 que regula las emisiones de CO2 provenientes del tubo de escape y su equivalencia en términos de rendimiento de combustible, aplicable a vehículos automotores nuevos de peso bruto vehicular de hasta 3.857 kilogramos. En esta norma se establece que para 2016, los automóviles deberán emitir desde 135 gCO2/km hasta 180 gCO2/km, dependiendo de su tamaño, y para camionetas de 163,6 gCO2/km hasta 227 gCO2/km dependiendo de su tamaño (DOF, 2013).  \label{Informe Final}

\paragraph{Japón}
De la misma manera, Japón establece un límite de consumo de combustible de 76,8 km/gal alcanzable para 2020 combustible en la política denominada “Top Runner” (AESS et al., 2011). \\ \label{Informe Final}

En el 2014, 27 de 34 países de la OECD modificaron la normatividad tributaria y realizaron exenciones para la adquisición de vehículos con alta eficiencia en el consumo de combustible y de baja emisión de contaminantes atmosféricos. Además, 18 países disminuyeron la tasa de cobro de impuestos sobre vehículos eléctricos o híbridos (Econcept - AEI, 2016). \label{Informe Final}

Una de las maneras para alcanzar la meta establecida de eficiencia en el consumo de combustible, es la implementación de tecnología en el vehículo. La aplicación de paquetes tecnológicos resulta viable, dado el desarrollo acelerado en las tecnologías para la reducción en el consumo de combustible y para la reducción de emisiones de gases de efecto invernadero y contaminantes criterio; además Colombia posee la ventaja de ser importante productor de refacciones para automóviles (Rivera Godoy et al., 2016). 

\section{GFEI}

La Global Fuel Economy Initiative (GFEI) se fundó en 2009 con el propósito de promover y apoyar la acción gubernamental para mejorar la eficiencia energética del parque vehicular ligero mundial de pasajeros. Originalmente el objetivo del GEFI era duplicar el ahorro de combustible de los nuevos vehículos de pasajeros en todo el mundo para 2030 (en relación con 2005) mediante mejoras continuas en la eficiencia de los motores de combustión interna más la introducción de vehículos de pasajeros eléctricos. Recientemente se ha ampliado este objetivo a una reducción del 50\% de las emisiones de CO2 por kilómetro de los nuevos vehículos de pasajeros para 2030 \cite{Prospects_GFEI}

 \begin{figure}[htbp]
   \centering
   \includesvg[width=0.5\textwidth]{figures/prospects-GFEI.svg}
    \caption{Eficiencia \textit{Tank-to-wheel} en carretera para los nuevos LDV \cite{Prospects_GFEI}}
    \label{fig:prospects-GFEI}
\end{figure}

Los socios del GFEI también establecen un nuevo objetivo de reducción de las emisiones de CO2 por kilómetro de los automóviles de pasajeros para 2050 del 90\% (también en relación con 2005). Para alcanzar este objetivo, el consumo de combustible de los motores de combustión tendrá que mejorar una media del 2,1\% anual entre 2020 y 2050, la fracción de ventas mundial de vehículos de pasajeros eléctricos tendrá que aumentar hasta el 35\% de las ventas en 2030 y el 86\% de las ventas en 2050, y la intensidad de carbono de la red eléctrica mundial tendrá que disminuir al menos un 90\% entre 2020 y 2050. \cite{Prospects_GFEI}

 \begin{figure}[htbp]
   \centering
   \includesvg[width=0.5\textwidth]{figures/prospectsco2-GFEI.svg}
    \caption{Emisiones $CO_2$ \textit{Well-to-wheel} en carretera para los nuevos LDV \cite{Prospects_GFEI}}
    \label{fig:prospects-GFEI}
\end{figure}

A nivel internacional el GFEI ha colaborado en la elaboración de líneas base y en recomendaciones e implementaciones de políticas que conrtibuyan a la incrementar la eficiencia energética en LDVs.

 \begin{figure}[htbp]
   \centering
   \includesvg[width=1\textwidth]{figures/country-GFEI.svg}
    \caption{Estatus del trabajo del GFEI en el Mundo \cite{GlobalStatus_GFEI}}
    \label{fig:statusworld-GFEI}
\end{figure}


En el caso de Colombia se ha elaborado la línea base y se han emitido recomendaciones por parte del GFEI y la ICCT, no obstante estas aún no han sido implementadas.

 \begin{figure}[htbp]
   \centering
   \includesvg[width=0.75\textwidth]{figures/status-GFEI.svg}
    \caption{Estatus de los objetivos del GFEI en Latinoamerica y el Caribe \cite{GlobalStatus_GFEI}}
    \label{fig:statuslatam-GFEI}
\end{figure}

\section{PCFV}

La Asociación para los Combustibles y Vehículos Limpios (PCFV) es la principal iniciativa mundial de carácter público-privado que promueve los combustibles y vehículos más limpios en los países en desarrollo y en los países en transición. Establecida en la Cumbre Mundial sobre el Desarrollo Sostenible de 2002, la PCFV reúne a 73 organizaciones que representan a países desarrollados y en desarrollo, a las industrias de combustibles y vehículos, a la sociedad civil y a los principales expertos mundiales en combustibles y vehículos más limpios. \cite{Vehicles_PCFV}

 \begin{figure}[htbp]
   \centering
   \includesvg[width=1\textwidth]{figures/status-PCFV.svg}
    \caption{Estatus del PCFV \cite{Vehicles_PCFV}}
    \label{fig:statuslatam-PCFV}
\end{figure}

\section{ICCT}

En términos generales gracias a datos preeliminares del ICCT podemos observar que en países donde se han implementado políticas para la eficiencia energética la tendencia de los Factores de Emisión (FE) es a la baja.

\begin{figure}[htbp]
   \centering
   \includesvg[width=1\textwidth]{figures/colombiacomp.svg}
    \caption{Comparación entre los Factores de Emisión de CO2 calculados para Colombia y los Factores de Emisión de CO2 reportados por el mercado automotriz en diferentes países. \cite{ICCT_Performance_2020}}
    \label{fig:linebaseclase}
\end{figure}

\section{Caso de estudio en la EU}

En la Unión Europea podemos observar mejoras cuantificables a partir de las normativas implementadas durante los últimos años. Los datos preeliminares de la \textit{European Environment Agency} (EEA) mostraron que los coches nuevos vendidos en la UE en 2020 tenían unas emisiones medias de $CO_2$ de $108g$ de $CO_2/km$, $14g/km$ menos que en 2019, medido sobre el Nuevo Ciclo de Conducción Europeo (NEDC). \cite{ICCT_Performance_2020} Gracias a los mecanismos de cumplimiento de la EU, el promedio de emisiones de $CO_2$ bajo el Cíclo NEDC se encuentra en $96 g/km$. El efecto de los mecanismo es contundente como podemos observar en \ref{fig:iccp-performance} y casi todos los fabricantes cumplieron sus objetivos de $CO_2$ para 2020.

 \begin{figure}[htbp]
   \centering
   \includesvg[width=1\textwidth]{figures/iccp-performance.svg}
    \caption{Valores Medios Históricos de las Emisiones de CO2 NEDC \cite{ICCT_Performance_2020}}
    \label{fig:iccp-performance}
\end{figure}